\documentclass[12pt]{tdtp}
\usepackage{hyperref}
\usepackage{tabularx,colortbl}
\usepackage{multirow}
\usepackage{listings}
\lstset{
	language=VHDL,
basicstyle=\tiny\ttfamily}
\definecolor{light-gray}{gray}{0.96}
\definecolor{pageheading-gray}{gray}{0.2}
\definecolor{dark-gray}{gray}{0.45}
\definecolor{dark-green}{rgb}{0.245,0.121,0.0}

\newcommand{\auteur}{Cedric Lemaitre}
\newcommand{\couriel}{c.lemaitre58@gmail.com}
\newcommand{\promo}{BScv}
\newcommand{\annee}{2017-2018}
\newcommand{\matiere}{Computer science}

\newcommand{\tdtp}{Lab \#1}
\renewcommand{\sujet}{Function, array, matrix}


\begin{document}
\titre
\textit{NB : use good practice for naming and write code \footnote{\url{https://google.github.io/styleguide/cppguide.html}}!!!}

%%%%%%%%%%%
\Exo

Write in C++ a class named "Cripto". This class take a string at construction. This class has as interne variable (private) 2 another strings.
Those strings will contain the result of 2 encryption methods

\begin{itemize}
	\item Caesar cipher \url{https://en.wikipedia.org/wiki/Caesar_cipher}
	\item Vigenere cipher \url{https://en.wikipedia.org/wiki/Vigen%C3%A8re_cipher}
\end{itemize}

Add also all get and set functions and the fountion to display results of encryption.

\Exo

Add methods to inverse the encryption to your "Cripto" classe.

\end{document}
